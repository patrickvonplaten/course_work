\documentclass[twoside,11pt,a4paper,english]{article}

% packages %%%%%%%%%%%%%%%%%%%%%%%%%%%%%%%%%%%%%%%%%%%%%%%%%%%%%%%%%%%%%%%%%%%%
\usepackage{graphicx,curves,float,rotating}

\usepackage{amsmath, amssymb, latexsym}  % math stuff
\usepackage{amsopn}                             % um mathe operatoren zu deklarieren
\usepackage[english]{babel}                     % otherwise use british or american
\usepackage{theorem}                            % instead of \usepackage{amsthm}
\usepackage{dcolumn}
\usepackage{hyperref}
\usepackage[]{algorithm2e}




% @ environment %%%%%%%%%%%%%%%%%%%%%%%%%%%%%%%%%%%%%%%%%%%%%%%%%%%%%%%%%%%%%%%%
\usepackage{xspace}                             % context sensitive space after macros
\makeatletter 
\DeclareRobustCommand\onedot{\futurelet\@let@token\@onedot}
\def\@onedot{\ifx\@let@token.\else.\null\fi\xspace}
\def\eg{{e.g}\onedot} \def\Eg{{E.g}\onedot}
\def\ie{{i.e}\onedot} \def\Ie{{I.e}\onedot}
\def\cf{{c.f}\onedot} \def\Cf{{C.f}\onedot}
\def\etc{{etc}\onedot} \def\vs{{vs}\onedot} 
\def\wrt{w.r.t\onedot} \def\dof{d.o.f\onedot}
\def\etal{{et al}\onedot}
\def\zB{z.B\onedot} \def\ZB{Z.B\onedot}
\def\dh{d.h\onedot} \def\Dh{D.h\onedot}
% %%%%%%%%%%%%%%%%%%%%%%%%%%%%%%%%%%%%%%%%%%%%%%%%%%%%%%%%%%%%%%%%%%%%%%%%%%%%%%%


%	Macros fuer neue Umgebungen
  

%%%%%%%%%%%%%%%%%%%%%%%%%%%%%%%%%%%%%%%%%%%%%%%%%%%%%%%%%%%%%%%%%%%%%%%%%%%%%%
\newcommand*{\Frac}[2]{\frac{\displaystyle #1}{\displaystyle #2}}
\newlength{\textwd}
\newlength{\oddsidemargintmp}
\newlength{\evensidemargintmp}
\newcommand*{\hspaceof}[2]{\settowidth{\textwd}{#1}\mbox{\hspace{#2\textwd}}}
\newlength{\textht}
\newcommand*{\vspaceof}[3]{\settoheight{\textht}{#1}\mbox{\raisebox{#2\textht}{#3}}}
\newcommand*{\PreserveBackslash}[1]{\let\temp=\\#1\let\\=\temp}

\newenvironment{deflist}[1][\quad]%
{  \begin{list}{}{%
      \renewcommand{\makelabel}[1]{\textbf{##1}\hfil}%
      \settowidth{\labelwidth}{\textbf{#1}}%
      \setlength{\leftmargin}{\labelwidth}
      \addtolength{\leftmargin}{\labelsep}}}
{  \end{list}}


\newenvironment{Quote}% Definition of Quote
{  \begin{list}{}{%
      \setlength{\rightmargin}{0pt}}
      \item[]\ignorespaces}
{\unskip\end{list}}


\theoremstyle{break}
\theorembodyfont{\itshape}	
\theoremheaderfont{\scshape}

\newtheorem{Cor}{Corollary}
\newtheorem{Def}{Definition}
%\newtheorem{Def}[Cor]{Definition}



\newcolumntype{.}{D{.}{.}{-1}}


\pagestyle{headings}
\textwidth 15cm
\textheight 23cm
\oddsidemargin 1cm
\evensidemargin 0cm
%\parindent 0mm



%%%%%%%%%%%%%%%%%%%%%%%%%%%%%%%%%%%%%%%%%%%%%%%%%%%%%%%%%%%%%%%%%%%%%%%%%%%%%%%
%
%
%       Jetzt geht's los
%
%
%%%%%%%%%%%%%%%%%%%%%%%%%%%%%%%%%%%%%%%%%%%%%%%%%%%%%%%%%%%%%%%%%%%%%%%%%%%%%%%
\begin{document}


%%%%%%%%%%%%%%%%%%%%%%%%%%%%%%%%%%%%%%%%%%%%%%%%%%%%%%%%%%%%%%%%%%%%%%%%%%%%%%%
%
%
%               Title
%
%
%%%%%%%%%%%%%%%%%%%%%%%%%%%%%%%%%%%%%%%%%%%%%%%%%%%%%%%%%%%%%%%%%%%%%%%%%%%%%%%
\pagestyle{empty}

\begin{center}

    Rheinisch-Westf\"alische Technische Hochschule Aachen \\
    Institute for Theoretical Information Technology \\
    Univ.-Prof. Dr. rer. nat. Rudolf Mathar \\[6ex]
    Seminar on Deep Learning - Methodologies and Applications - SS2018\\[12ex]                          % auch Seminar Titel und Datum �ndern!!!
   
    \LARGE
    \textbf{Deep Learning for the Physical Layer} \\[6ex]
    \textit{Patrick von Platen} \\[6ex]
    \Large
    Matrikelnummer 331 430 \\[6ex]
    Datum des Vortrages

    \vfill
    \Large Betreuer: Johannes Schmitz 
	    
\end{center}

\newpage
\ 
\newpage

%%%%%%%%%%%%%%%%%%%%%%%%%%%%%%%%%%%%%%%%%%%%%%%%%%%%%%%%%%%%%%%%%%%%%%%%%%%%%%%
%
%
%               Inhaltsverzeichnis / Tabellenverzeichnis / Abbildungsverz.
%
%
%
%%%%%%%%%%%%%%%%%%%%%%%%%%%%%%%%%%%%%%%%%%%%%%%%%%%%%%%%%%%%%%%%%%%%%%%%%%%%%%%
\pagestyle{headings}
\tableofcontents
\listoftables
\listoffigures
\newpage
\pagestyle{empty}
\ 
\newpage
\pagestyle{headings}

\section{Motivation} % (fold)
\label{sec:introduction}

Concentrate only on complete end-to-end autoencoder network. Not enough research on 
communication system using one sided neural networks.

Introduce reader to the topic. 
Explain motivation: 
\begin{itemize}
	\item so far every part of communication layer has to be optimized by itself
	\item very hard to find mathematical optimization for complete end - to -end system
	\item use autoencoder for end-to-end usage
\end{itemize}

\cite{DBLP:journals/corr/OSheaH17}
\cite{2018arXiv180303145O}
\cite{8262721}
\cite{2018ISTSP..12..132D}

% section introduction (end)

\section{Conventional communication systems} 
\label{sec:communication_systems}

Describe here how conventional communication systems look 
like. 

\subsection{Structure of system}


\subsubsection{Transmitter}

Role of transmitter 


\subsubsection{Channel}

What is the actual channel 

\subsubsection{Receiver}

Role of receiver 
(put receiver and transmitter together)

% section physical_layer

\subsection{Advantages of current system}

In what ways are current communication systems already very 
efficient? 

\section{Applying deep learning to communication systems}

Explain the motivation of using deep learning methods to model 
end-to-end communication systems.
In which cases could deep learning be applicable for communication systems

\begin{itemize}
	\item possible efficiency improvements (or not)
	\item great adaptive properties to model communication systems in 
		unusual conditions (non-gaussian noise,...)
\end{itemize}

\subsection{Deep neural network basics}

Explain basics of DNN: MLP, every function can be modeled by DNN, 
activation functions, backpropagation,...

\subsection{Convolutional Layers in NN}
Explain how Conv. layers work and it's relation to signal analysis/
processing (applying conv layer to time signal is similar to applying 
fourier transform to it)
Why are they useful in this case?

\subsection{Autoencoder}

Explain motivation and why it works in case of communication systems.
Describe auto-encoder network in detail and show it's obvious relation to 
the physical layer in communication systems.

\subsection{Synchronization of information transfer}

How can neural networks overcome the problem of synchronization? 
What methods are used? (Start sending bit stream?)

\subsection{Training of autoencoder systems}

How is training be done?

\section{State-of-the-art deep learning methods for communication systems}

Give examples of system and their architecture

\subsection{Example system}

\subsection{Comparison deep learing methods and conventional methods}

Compare current ``state-of-the-art'' to system using deep learning.

\subsubsection{Performance measurement}

Define and introduce Signal-to-noise ration

\subsubsection{Results using deep learning methods}

State attained results. 

\section{Conclusion} % (fold)
\label{sec:conclusion}

Conclude by trying to answer the question, whether it is useful 
to apply deep learning methods for communication systems (or under which 
circumstances it might be useful)

% section conclusion (end)
\newpage

\addcontentsline{toc}{section}{References}
\bibliographystyle{plain}
\bibliography{vonPlaten_report}

\end{document}
