\documentclass[twoside,11pt,a4paper,english]{article}

% packages %%%%%%%%%%%%%%%%%%%%%%%%%%%%%%%%%%%%%%%%%%%%%%%%%%%%%%%%%%%%%%%%%%%%
\usepackage{graphicx,curves,float,rotating}

\usepackage{amsmath, amssymb, latexsym}  % math stuff
\usepackage{amsopn}                             % um mathe operatoren zu deklarieren
\usepackage[english]{babel}                     % otherwise use british or american
\usepackage{theorem}                            % instead of \usepackage{amsthm}
\usepackage{dcolumn}
\usepackage{hyperref}
\usepackage[]{algorithm2e}




% @ environment %%%%%%%%%%%%%%%%%%%%%%%%%%%%%%%%%%%%%%%%%%%%%%%%%%%%%%%%%%%%%%%%
\usepackage{xspace}                             % context sensitive space after macros
\makeatletter 
\DeclareRobustCommand\onedot{\futurelet\@let@token\@onedot}
\def\@onedot{\ifx\@let@token.\else.\null\fi\xspace}
\def\eg{{e.g}\onedot} \def\Eg{{E.g}\onedot}
\def\ie{{i.e}\onedot} \def\Ie{{I.e}\onedot}
\def\cf{{c.f}\onedot} \def\Cf{{C.f}\onedot}
\def\etc{{etc}\onedot} \def\vs{{vs}\onedot} 
\def\wrt{w.r.t\onedot} \def\dof{d.o.f\onedot}
\def\etal{{et al}\onedot}
\def\zB{z.B\onedot} \def\ZB{Z.B\onedot}
\def\dh{d.h\onedot} \def\Dh{D.h\onedot}
% %%%%%%%%%%%%%%%%%%%%%%%%%%%%%%%%%%%%%%%%%%%%%%%%%%%%%%%%%%%%%%%%%%%%%%%%%%%%%%%


%	Macros fuer neue Umgebungen
  

%%%%%%%%%%%%%%%%%%%%%%%%%%%%%%%%%%%%%%%%%%%%%%%%%%%%%%%%%%%%%%%%%%%%%%%%%%%%%%
\newcommand*{\Frac}[2]{\frac{\displaystyle #1}{\displaystyle #2}}
\newlength{\textwd}
\newlength{\oddsidemargintmp}
\newlength{\evensidemargintmp}
\newcommand*{\hspaceof}[2]{\settowidth{\textwd}{#1}\mbox{\hspace{#2\textwd}}}
\newlength{\textht}
\newcommand*{\vspaceof}[3]{\settoheight{\textht}{#1}\mbox{\raisebox{#2\textht}{#3}}}
\newcommand*{\PreserveBackslash}[1]{\let\temp=\\#1\let\\=\temp}

\newenvironment{deflist}[1][\quad]%
{  \begin{list}{}{%
      \renewcommand{\makelabel}[1]{\textbf{##1}\hfil}%
      \settowidth{\labelwidth}{\textbf{#1}}%
      \setlength{\leftmargin}{\labelwidth}
      \addtolength{\leftmargin}{\labelsep}}}
{  \end{list}}


\newenvironment{Quote}% Definition of Quote
{  \begin{list}{}{%
      \setlength{\rightmargin}{0pt}}
      \item[]\ignorespaces}
{\unskip\end{list}}


\theoremstyle{break}
\theorembodyfont{\itshape}	
\theoremheaderfont{\scshape}

\newtheorem{Cor}{Corollary}
\newtheorem{Def}{Definition}
%\newtheorem{Def}[Cor]{Definition}



\newcolumntype{.}{D{.}{.}{-1}}


\pagestyle{headings}
\textwidth 15cm
\textheight 23cm
\oddsidemargin 1cm
\evensidemargin 0cm
%\parindent 0mm



%%%%%%%%%%%%%%%%%%%%%%%%%%%%%%%%%%%%%%%%%%%%%%%%%%%%%%%%%%%%%%%%%%%%%%%%%%%%%%%
%
%
%       Jetzt geht's los
%
%
%%%%%%%%%%%%%%%%%%%%%%%%%%%%%%%%%%%%%%%%%%%%%%%%%%%%%%%%%%%%%%%%%%%%%%%%%%%%%%%
\begin{document}


%%%%%%%%%%%%%%%%%%%%%%%%%%%%%%%%%%%%%%%%%%%%%%%%%%%%%%%%%%%%%%%%%%%%%%%%%%%%%%%
%
%
%               Title
%
%
%%%%%%%%%%%%%%%%%%%%%%%%%%%%%%%%%%%%%%%%%%%%%%%%%%%%%%%%%%%%%%%%%%%%%%%%%%%%%%%
\pagestyle{empty}

\begin{center}

    Rheinisch-Westf\"alische Technische Hochschule Aachen \\
    Institute for Theoretical Information Technology \\
    Univ.-Prof. Dr. rer. nat. Rudolf Mathar \\[6ex]
    Seminar on Deep Learning - Methodologies and Applications - SS2018\\[12ex]                          % auch Seminar Titel und Datum �ndern!!!
   
    \LARGE
    \textbf{Deep Learning for the Physical Layer} \\[6ex]
    \textit{Patrick von Platen} \\[6ex]
    \Large
    Matrikelnummer 331 430 \\[6ex]
    Datum des Vortrages

    \vfill
    \Large Betreuer: Johannes Schmitz 
	    
\end{center}

\newpage
\ 
\newpage

%%%%%%%%%%%%%%%%%%%%%%%%%%%%%%%%%%%%%%%%%%%%%%%%%%%%%%%%%%%%%%%%%%%%%%%%%%%%%%%
%
%
%               Inhaltsverzeichnis / Tabellenverzeichnis / Abbildungsverz.
%
%
%
%%%%%%%%%%%%%%%%%%%%%%%%%%%%%%%%%%%%%%%%%%%%%%%%%%%%%%%%%%%%%%%%%%%%%%%%%%%%%%%
\pagestyle{headings}
\tableofcontents
\listoftables
\listoffigures
\newpage
\pagestyle{empty}
\ 
\newpage
\pagestyle{headings}

\section{Motivation} % (fold)
\label{sec:introduction}

Concentrate only on complete end-to-end autoencoder network. Not enough research on 
communication system using one sided neural networks.

Introduce reader to the topic. 
Explain motivation: 
\begin{itemize}
	\item so far every part of communication layer has to be optimized by itself
	\item very hard to find mathematical optimization for complete end - to -end system
	\item use autoencoder for end-to-end usage
\end{itemize}


\cite{DBLP:journals/corr/OSheaH17}
\cite{2018arXiv180303145O}
\cite{8262721}
\cite{2018ISTSP..12..132D}
\cite{synch1}
\cite{synchAttention}

% section introduction (end)

\section{Conventional Communication Systems} 
\label{sec:communication_systems}

This section aims at explaining what the so-called physical layer is actually made of 
by describes it function and structure in detail. Later in this section, an example system will be explained and used for performance comparasion with communcation systems applying deep learning methods \ref{sec:applying_deep_learning_to_communication_systems}.

In this report, we use the word \emph{communication system} as the \emph{Physical Layer} of the well-know \emph{OSI Model} as a \emph{communication system}.
As it is defined in \cite{osimodel}, ``the physical layer manages the reception and transmission of the unstructured raw bit stream over a physical medium''. 
The physical layer therefore comprises transforming the raw bit stream to electromagnetic waves that will be transmitted by an antenna at the transmitter, receiving the 
electromagnetic waves via an antenna at the receiver and using them to recover the raw bit stream. 
In the following the connection between the antenna of the transmitter and the antenna of
the receiver will be defined as the \emph{channel}.

\subsection{Structure of the Communication System}

The role of the communication system is too two-fold:

\begin{itemize}
	\item Efficient Transmitting: The raw bit stream should be converted in such a 
		way that it can be transmitted at a high bit rate in terms of the \emph{Shannon information theory} \cite{Shannon:2001:MTC:584091.584093}
	\item Reliable Delivery: In case of errors to due noise in the channel, the 
		communication system must be able to do error detection and error correction. 
\end{itemize}

This can be achieved by applying sophisticated encoding and decoding schemes also just called ``codes'' to the raw bit stream. Such a ``code'' would for example be the so-called
``Turbo-Codes'' \cite{Berrou93nearshannon} that use a special encoding and decoding schemeto reach ``near Shannon limit error-correcting coding and decoding'' meaning that the 
maximum rate at which information can be transmitted over a channel of a specified 
bandwidth in the presence of noise, being the \emph{Shannon rate}, can nearly be archieved while the methods allows for error correction.

Now let's look at the three parts of a communication system, being the \emph{transmitter,
the channel and the receiver}

\subsubsection{Transmitter}


\subsubsection{Channel}


\subsubsection{Receiver}

\subsection{Convential System Design}

What methods and algorithms are the most used and state-of-the-art 
nowadays?

\section{Deep Learning Basics}%
\label{sec:deep_learning_basics}

Intro into the most important deep learning basics needed to 
model the physical layer. 
Always mention how it can be applied for the physical layer!

\subsection{Deep Neural Network}%
\label{sub:deep_neural_network}

Explain multi layer perceptron. General idea of neural networks

\subsection{Convolutional Layers in NN}
Explain how Conv. layers work and it's relation to signal analysis/
processing (applying conv layer to time signal is similar to applying 
fourier transform to it)
Why are they useful in this case?


\subsection{Autoencoder}

Explain motivation and why it works in case of communication systems.
Describe auto-encoder network in detail and show it's obvious relation to 
the physical layer in communication systems.

\subsection{Generative Adversarial Networks}%
\label{sub:generative_adversarial_networks}

Explain Generative Adversarial Networks and how they can be used to 
model the channel with its noise.
This should be explained in detail.


\section{Applying deep learning to communication systems}%
\label{sec:applying_deep_learning_to_communication_systems}



Explain the motivation of using deep learning methods to model 
end-to-end communication systems.
In which cases could deep learning be applicable for communication systems

\begin{itemize}
	\item possible efficiency improvements (or not)
	\item great adaptive properties to model communication systems in 
		unusual conditions (non-gaussian noise,...)
\end{itemize}

Also explain problems: Network is trained on assumptions about the channel 
distribution that might not be true $\to$ Generative Adversarial Networks try 
to solve this problem in some cases

\subsection{Artificial neural network supported end-to-end systems}%
\label{sub:artificial_neural_network_supported_end_to_end_systems}

Describe here two approaches: the base approach where the channel 
is just modeled by a gaussian noise layer \cite{2018ISTSP..12..132D} and \cite{DBLP:journals/corr/OSheaH17}
and the more sophisticated approach using Generative Adversarial Networks \cite{2018arXiv180506350O} and \cite{2018arXiv180303145O}.
Describe architectures and training in detail.

\subsubsection{Basic Deep Learning based Communication system}%
\label{ssub:basic_deep_learning_based_communication_system}

\subsubsection{Generative Adversarial Networks based Communicaton system}%
\label{ssub:generative_adversarial_networks_based_communicaton_system}

\subsection{Comparison deep learing methods and conventional methods}

Compare current ``state-of-the-art'' to system using deep learning.

\subsubsection{Performance measurement}

Define and introduce Signal-to-noise ration

\subsubsection{Results using deep learning methods}

State attained results. 

\section{Conclusion} % (fold)
\label{sec:conclusion}

Conclude by trying to answer the question, whether it is useful 
to apply deep learning methods for communication systems (or under which 
circumstances it might be useful)

% section conclusion (end)
\newpage

\addcontentsline{toc}{section}{References}
\bibliographystyle{plain}
\bibliography{vonPlaten_report}

\end{document}
