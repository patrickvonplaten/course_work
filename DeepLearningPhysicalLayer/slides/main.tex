\documentclass[xcolor=table,mathserif,9pt]{beamer}    % ,handout
% colortbl only defines \rowcolor for a single row. xcolor extends this to multiple rows.

\usetheme{Aachen}
\usepackage[english]{babel}
\usepackage[utf8]{inputenc}
%\usepackage{tikz-dependency}
%\usepackage{chronology}
\usepackage{array, booktabs}
\newcommand{\foo}{\makebox[0pt]{\textbullet}\hskip+0.5pt\vrule width 5pt\hspace{\labelsep}}
\usepackage{multirow}
\usepackage{textcomp}
\usepackage{media9}
\usepackage{csquotes}
\usepackage{amsmath}
\usepackage[]{algorithm2e}
\usepackage{lipsum}
\usepackage{multicol}
\usepackage{hyperref}
\setlength{\columnsep}{1cm}
%%%%%%%%%%%%%%%%%%%%%%%%%%%%%%%%%%%%%%%%%%%%%%%%%%%%%%%%%%%%%%%%%%%%%%
% tables
\usepackage{multirow,array,tabularx,rotating}
\usepackage{booktabs}
\usepackage{tabularx}

% math
\usepackage{amsmath,amsthm, amssymb, latexsym, xspace}
%\usepackage{bbold}
\usefonttheme[onlymath]{serif}
\boldmath


% misc
\usepackage{subfigure}
\usepackage{wasysym}
\usepackage{nameref}
\usepackage{xcolor}
\usepackage{romannum}
% declare the path(s) where your graphic files are
\graphicspath{{./nlu/}{./g2p/}{./smt/}}
% and their extensions so you won't have to specify these with
% every instance of \includegraphics
\DeclareGraphicsExtensions{.pdf,.jpeg,.png}


%figures
\usepackage{tikz}
\usepackage{minibox}
% change the graphic extentions
%\usepackage{ifpdf}
%\ifpdf
%  \DeclareGraphicsExtensions{.pdf,.png,.jpg}
%\else
%  \DeclareGraphicsExtensions{.eps}
%\fi

\usepackage[np,autolanguage]{numprint}
\nprounddigits{1}

% helpers
\newcommand{\argmin}{\operatornamewithlimits{argmin}}
\newcommand{\argmax}{\operatornamewithlimits{argmax}}
\newcommand{\sign}{\operatornamewithlimits{sign}}
\newcommand{\Eqn}{Equation}
\newcommand{\Eqns}{Equations}
\newcommand{\Fig}{Figure}
\newcommand{\Figs}{Figures}
\newcommand{\Tab}{Table}
\newcommand{\Sec}{Section}
\def\example{{\textit{}{e.g.}}\xspace}
\def\cad{{\textit{}{i.e.}}\xspace}
\def\etc{{\textit{etc.}}\xspace}
\def\apriori{{\textit{a priori}}\xspace}

\newcommand{\NetTalk}{NETtalk\xspace}
\newcommand{\Celex}{Celex\xspace}
\newcommand{\Pronlex}{Pronlex\xspace}
\newcommand{\gtp}{G2P}
\newcommand{\Seg}{\mathbb{S}}
\newcommand\BLEU{\textsc{Bleu}\xspace}
\newcommand\TER{\textsc{Ter}\xspace}
\newcommand\CTER{\textsc{CTer}\xspace}

\newcommand\AER{\textsc{Aer}\xspace}
\newcommand\SAER{\textsc{Saer}\xspace}
\newcommand{\GIZA}{{GIZA\nolinebreak[4]\hspace{-.025em}\raisebox{.2ex}{\small\bf++}}\xspace}

\newcommand{\todo}[1]{\colorbox{yellow}{#1}\xspace}
\newcommand{\CITE}{\colorbox{yellow}{CITE}\xspace}
\newcommand{\REF}{\colorbox{yellow}{REF}\xspace}
\newcommand{\EQ}{\colorbox{yellow}{REF}\xspace}
\newcommand{\NUMBER}{\colorbox{yellow}{NUMBER}\xspace}
\newcommand{\Align}{\mathbb{A}}
%\renewcommand{\emph}[1]{\textcolor{i6blue}{#1}}


\newcommand{\aind}[1]{\hspace*{#1ex}}
\newcommand{\gray}[1]{\textcolor{gray}{#1}}

\newcommand*{\sumw}{\ensuremath{\operatornamewithlimits{sum}}\xspace}
\newcommand*{\nbest}{\ensuremath{\operatornamewithlimits{nbest}}\xspace}

% Define box and box title style
\usepackage{tikz}
\usetikzlibrary{shapes,positioning,fit}
\tikzstyle{bubble} = [ draw=blue, rectangle, rounded corners, inner sep=3pt, inner ysep=
10 pt]
\tikzstyle{fancytitle} =[fill=white, text=black]
\newenvironment{bubble}[3]{%
  \begin{tikzpicture}[transform shape, baseline=-0.5 cm]
  \def\bubbletitle{#1}%
  \node [bubble] (box)\bgroup
  \begin{minipage}[t][#3]{#2}%
}{%
  \end{minipage}%
  \egroup;
  \node[fancytitle] at (box.north) {\bubbletitle};
  \end{tikzpicture}%
}

\newenvironment{bubble*}[2]{%
  \begin{tikzpicture}[transform shape]
  \node [bubble] (box)\bgroup
  \begin{minipage}[t][#2]{#1}%
}{%
  \end{minipage}%
  \egroup;
  \end{tikzpicture}%
}

% CUSTOM STUFF %%%%%%%%%%%%%%%%%
%\usepackage{neuralnetworks}


% Rounding commands
\def\roundpositionii{1}
\newcommand{\rdmii}[1]{\edef\rounded{0}\FPeval\rounded{round(#1,\roundpositionii)}\rounded}
\def\roundpositioni{1}
\newcommand{\rdmi}[1]{\edef\rounded{0}\FPeval\rounded{round(#1,\roundpositioni)}\rounded}
\def\roundposition{0}
\newcommand{\rdm}[1]{\edef\rounded{0}\FPeval\rounded{round(#1,\roundposition)}\rounded}

\npstyleenglish

\def\roundposition{1}
\def\roundpositiont{3}
\edef\rounded{0}
\newcommand{\rd}[1]{\ifthenelse{\equal{#1}{--}}{--}{\edef\rounded{0}\FPeval\rounded{round(#1,\roundposition)}\rounded}}
\newcommand{\rdmt}[1]{\edef\rounded{0}\FPeval\rounded{round(#1,\roundpositiont)}\rounded}
\newcommand{\rp}[1]{%
  \FPset{\per}{#1}%
  %\FPmul{\per}{\per}{100}%
  \FPround{\per}{\per}{1}%
  \numprint{\per}%
}%


\setbeamercovered{transparent}

\DeclareMathOperator*{\sigmoid}{sigmoid}

\newcommand\T{\rule{0pt}{2.2ex}}       % Top strut
\usepackage{etoolbox}
\pretocmd{\section}{\addtocontents{toc}{\vspace{-20pt}}}{}{}

%Table stuff
\newcommand{\tablesize}{}
\newcommand{\abovetable}{\vspace{0.6cm}}
\newcommand{\belowtable}{\vspace{0.13cm}}


% Get fond size for system combination picture right
\usepackage{fix-cm}    
\makeatletter
\newcommand\SyscomFontSize{\@setfontsize\small{6}{60}}
\makeatother   

%%%%%%%%%%%%%%%%%%%%%%%%%%%%%%%%%

\newcommand{\backupbegin}{
   \newcounter{finalframe}
   \setcounter{finalframe}{\value{framenumber}}
}
\newcommand{\backupend}{
   \setcounter{framenumber}{\value{finalframe}}
}

\newcommand{\stoptocwriting}{%
  \addtocontents{toc}{\protect\setcounter{tocdepth}{-5}}}
\newcommand{\resumetocwriting}{%
  \addtocontents{toc}{\protect\setcounter{tocdepth}{2}}}

%%%%%%%%%%%%%%%%%%%%%%%%%%%%%%%%%%%%%%%%%%%%%%%%%%%%%%%%%%%%%%%%%%%%%%  
%%%%%%%%%%%%%%%%%%%%%%%%%%%%%%%%%%%%%%%%%%%%%%%%%%%%%%%%%%%%%%%%%%%%%%

\renewcommand*{\email}{\url{patrick.platen@rwth-aachen.de}}  
% all email address(es) of the authors (used for \TitlePage)

\title[Seminar]{Deep Learning for the Physical Layer}
%\subtitle{Presentation Subtitle} % (optional)
%\setbeamertemplate{section in toc}[sections numbered] 
%\setbeamertemplate{subsection in toc}[subsections numbered] 
\setbeamertemplate{navigation symbols}{} %disable {, dotted}navigation bar

%% author and in []: shortauthorChange the scale of beamer from 3.5 to 1
\author[Patrick von Platen]{Patrick von Platen}
% - Use the \inst{?} command only if the author7s have different
%   affiliation.
\institute[RWTH Aachen University] % (optional, but mostly needed)
{
%  \inst{1}%
  \strut Institute for Theoretical Information Technology\\
  \strut Univ.-Prof. Dr. rer. nat. Rudolf Mathar%\\
  %\strut {\tt lehnen@cs.rwth-aachen.de}
}
% - Use the \inst command only if there are several affiliations.
% - Keep it simple, no one is interested in your street address.

\date[19/7/2018]{July 19th, 2018}

%%%%%%%%%%%%%%%%%%%%%%%%%%%%%%%%%%%%%%%%%%%%%%%%%%%%%%%%%%%%%%%%%%%%%%shit
% will be set into the PDF document summary
\hypersetup{
  pdftitle={\inserttitle}, 
  pdfauthor={\insertauthor}, 
  bookmarksdepth=subsubsection,  
  % enable automatic page transitions: for endless loop edit in
  % acrobat reader -> preferences -> full screen -> after every X
  % seconds and after last page
  %pdfpageduration = 2, 
  % pdfpagetransition = {Glitter /Di 315 /D 5}  
  % pdfpagetransition = {Box /M /O /D 1},
} 
%%%%%%%%%%%%%%%%%%%%%%%%%%%%%%%%%%%%%%%%%%%%%%%%%%%%%%%%%%%%%%%%%%%%%%
%%%%%%%%%%%%%%%%%%%%%%%%%%%%%%%%%%%%%%%%%%%%%%%%%%%%%%%%%%%%%%%%%%%%%%

%%%%%%%%%%%%%%%%%%%%%%%%%%%%%%%%%%%%%%%%%%%%%%%%%%%%%%%%%%%%%%%%%%%%%%
%%%%%%%%%%%%%%%%%%%%%%%%%%%%%%%%%%%%%%%%%%%%%%%%%%%%%%%%%%%%%%%%%%%%%%
\begin{document}

%%%%%%%%%%%%%%%%%%%%%%%%%%%%%%%%%%%%%%%%%%%%%%%%%%%%%%%%%%%%%%%%%%%%%%
\begin{frame}[label=titlepage]
  \titlepage
\end{frame}

\begin{frame}
	\frametitle{Outline}
	\tableofcontents
        %\tableofcontents[currentsection, subsectionstyle=show/show/hide]
\end{frame}


%%%%%%%%%%%%%%%%%%%%%%%%%%%%%%%%%%%%%%%%%%%%%%%%%%%%%%%%%%%%%%%%%%%%%%

\section{Literature}%
\label{sec:literature}


\begin{frame}{Literature}
\begin{description}
\item [Z. Chen:] Single Channel auditory source separation with neural network. {\em thesis 2017}.
  \begin{itemize}
  \item Output permutation problem  
  \item Output dimension problem 
  \item Deep clustering
  \end{itemize}
\end{description}
\end{frame}

\section{Introduction}%
\label{sec:introduction}
\begin{frame}{Introduction}

\begin{itemize}
	\item Apply deep learning to the physical layer of communcation systems
	\item Explain motivation
\end{itemize}

\textbf{Deep Learning for the Physical Layer}

\begin{figure}[htpb]
	\centering
	\includegraphics[width=.6\textwidth]{images/blocksComSystems.png}
	\caption{Block diagram for conventional physical layer}
\end{figure}

\end{frame}

\section{Source and Channel Encoding}%
\begin{frame}{Source and Channel Encoding}

explain Source and Channel Encoding
\end{frame}

\section{Physical Channel and Modulation}%
\begin{frame}{IQ - Modulation}

exlapin Channel
explain I/Q Modulation use picture

\end{frame}

\section{Autoencoder as Communication System}
\begin{frame}{Autoencoder as Communication System (1)}

Intro -> show pictures
Explain structure

\end{frame}

\begin{frame}{Autoencoder as Communication System (2)}

Mathematic's
Training with gradient descent

\end{frame}

\begin{frame}{Autoencoder as Communication System (3)}

Learned Modulation Schemes and results
\end{frame}

\section{Problem when using Neural Networks for Communication Systems}
\begin{frame}{Problem when using Neural Networks for Communication Systems}

- Modelling of the Physical channel not always accurate
- Synchronization

\end{frame}

\section{Generative Adversial Networks for Communication Systems}
\begin{frame}{Generative Adversial Networks for Communication Systems (1)}

Idea
Picture
General Idea of Generative Adversial Networks

\end{frame}

\begin{frame}{Generative Adversial Networks for Communication Systems (2)}

Training

\end{frame}

\section{Synchronization in Autoencoders}
\begin{frame}{Synchronization in Autoencoders (1)}

show first picture

\end{frame}

\begin{frame}{Synchronization in Autoencoders (2)}

show second picture

\end{frame}

\section{Conclusion}
\begin{frame}{Conclusion}

draw conclusion

\end{frame}

\begin{frame}[label=finalSlide]
  \label{LastPage}%
  \begin{center}
    \vfill
    {\Large
    \textcolor{i6bluedark}{Thank you for your attention}
    }
%     \vfill
%     \inserttitle
    \vfill
    {\insertauthor}

    \vspace{10mm}
    \url{patrick.platen@rwth-aachen.de}
  \end{center}
\end{frame}

\nocite{*}

\stoptocwriting
\section{Appendix}
\resumetocwriting

%\backupbegin


%%%%%%%%%%%%%%%%%%%%%%%%%%%%%%%%%%%%%%%%%%%%%%%%%%%%%%%%%%%%%%%%%%%%%%%%%%%%%%%%
\section*{Appendix}



%%%%%%%%%%%%%%%%%%%%%%%%%%%%%%%%%%%%%%%%%%%%%%%%%%%%%%%%%%%%%

\begin{frame}[allowframebreaks]
  \centerline{Reference}
 %\bibliographystyle{ieeetr}
 \bibliographystyle{i6bibstyle}
 \bibliography{references}
\end{frame}

%\backupend

\end{document}

