\section*{Exercise 3} % (fold)
\label{sec:exercise_3}

\subsection*{a)} % (fold)
\label{sub:a}

\textbf{First of all, it has to be noted that the given ``distant measure'' is in fact no distant measure because 
it does not fullfill the triange equality, f.e for $D=1, K=1$
\[
d(7,2) = 25 > d(7,5) + d(5,3) = 13
\]
We will nevertheless verify whether we have shift invariance and translation invariance}.

Let x be a column vector
$x = \begin{bmatrix}
x_{1}\\ 
x_{2}\\ 
..\\ 
x_{d} \\ 
\end{bmatrix}$
and y be a column vector
$y = \begin{bmatrix}
y_{1}\\ 
y_{2}\\ 
..\\ 
y_{d} \\ 
\end{bmatrix}$.

Then $K^{-1}$, the inverse of the diagonalized covariance matrix $C$ can be denoted as, 
$K^{-1}= diag(var(x_{1}), var(x_{2}), ..., var(x_{n})^{-1})^{-1} = diag(\frac{1}{var(x_{1})}, \frac{1}{var(x_{2})}, ..., \frac{1}{var(x_{n})})$ \\

Finally, $d(x,y): \mathbb{R}^D \times \mathbb{R}^D \to \mathbb{R}$ can be denoted as following: \newline 

\begin{align*}
d(x,y) &=\begin{bmatrix}
x_{1}-y_{1}	& x_{2}-y_{2} & ... & x_{d}-y_{d} \\ 
\end{bmatrix}
\begin{bmatrix}
\frac{1}{var(x_{1})} & 0 &  ...& 0 \\ 
0&  \frac{1}{var(x_{2})}& ... &0 \\ 
..& ... &  ..& 0\\ 
0& 0& 0 & \frac{1}{var(x_{d})} \\ 
\end{bmatrix}
\begin{bmatrix}
x_{1}-y_{1}\\ 
x_{2}-y_{2}\\ 
..\\ 
x_{d}-y_{d}\\
\end{bmatrix} \\
&= \sum_{i=1}^{D}\frac{(x_i - y_i)^2}{var(x_i)} \\
\end{align*}


% \begin{bmatrix}
% \frac{(x_{1}-y_{1})^{2}}{var(X_{1})} & 0 &  ...& 0\\
% 0&  \frac{(x_{2}-y_{2})^{2}}{var(X_2)}& ... &0 \\
% ..& ... &  ..& 0\\
% 0& 0& 0 & \frac{(x_d-y_d)^2}{var(X_d)} \\
% \end{bmatrix}


%Tried to simplify...
%In other words,\\
%distance measure
%$d(x,y)=\sum_{i=1}^{d}(\frac{x_{i}-y_{i%}}{sd_(i)})^{2}$

\subsubsection*{shift invariant} % (fold)
\label{ssub:shift invariant}
d(x,y) is shift invariant i.f.f. \\ 
$d({x}',{y}') := d(x-a,y-a) = d(x,y)$  
when 
let ${x}' \rightarrow x + a$ and ${y}' \rightarrow y + a$ where $x, \in R^{D}$. \newline 
\begin{align}
d(x-a,y-a) 
&=\sum_{i=1}^{D}\frac{(x_i - a - (y_i - a))^2}{var(x_i-a)} \\
\text{(using that var of constant is zero)} &=\sum_{i=1}^{D}\frac{(x_i - y_i)^2}{var(x_i)} \\
&= d(x,y) \\
\end{align}

Thus, d(x,y) is invariant under translation.

% subsubsection shift invariant (end)

\subsubsection*{scale invariant} % (fold)
\label{ssub:scale invariant}
d(x,y) is scale invariant i.f.f. \newline
$d({x}',{y}') := d(x,y)$
where  $x \rightarrow {x}' =
\begin{bmatrix}
	c_{1}x_{1}\\ 
	c_{2}x_{2}\\ 
	..\\ 
	c_{d}x_{d} \\ 
\end{bmatrix}$
and 
$y \rightarrow {y}' = 
\begin{bmatrix}
	c_{1}y_{1}\\ 
	c_{2}y_{2}\\ 
	..\\ 
	c_{d}y_{d} \\ 
\end{bmatrix}$
with $c_{d} \in \reals$. \newline

\begin{align}
d({x}',{y}') &= \sum_{i=1}^{D}\frac{({c_i}x_i - {c_i}y_i)^2}{var({c_i}x_i)}\\
&= \sum_{i=1}^{D}\frac{{c_i}^2(x_i - y_i)^2}{{c_i}^2var(x_i)} \\
&= \sum_{i=1}^{D}\frac{(x_i - y_i)^2}{var(x_i)}\frac{{c_i}^2}{{c_i}^2} \\
&= \sum_{i=1}^{D}\frac{(x_i - y_i)^2}{var(x_i)} \\
&= d(x,y) \\
\end{align}

 % &=\begin{bmatrix}
% \frac{(c_{1} \cdot (x_{1}-y_{1}))^{2}}{var(c_{2} \cdot X_{1})} & 0 &  ...& 0\\
% 0&  \frac{(c_{2} \cdot (x_{2}-y_{2}))^{2}}{var(c_{2} \cdot X_2)}& ... &0 \\
% ..& ... &  ..& 0\\
% 0& 0& 0 & \frac{(c_{d} \cdot (x_{d}-y_{d}))^{2}}{var(c_{d} \cdot X_d)} \\
% \end{bmatrix}\\
% &= \begin{bmatrix}
% \% frac{c_{1}^{2}}{c_{1}^{2}}\cdot  \frac{(x_{1}-y_{1})^{2}}{var(X_{1})} & 0 &  ...& 0\\
% 0&  \frac{c_{2}^{2}}{c_{2}^{2}}\cdot  \frac{(x_{2}-y_{2})^{2}}{var(X_2)}& ... &0 \\
% ..& ... &  ..& 0\\
% 0& 0& 0 & \frac{c_{d}^{2}}{c_{d}^{2}}\cdot \frac{(x_d-y_d)^2}{var(X_d)} \\
% \end{bmatrix}\\
% &=\begin{bmatrix}
% \frac{(x_{1}-y_{1})^{2}}{var(X_{1})} & 0 &  ...& 0\\
% 0&  \frac{(x_{2}-y_{2})^{2}}{var(X_2)}& ... &0 \\
% ..& ... &  ..& 0\\
% 0& 0& 0 & \frac{(x_d-y_d)^2}{var(X_d)} \\
% \end{bmatrix}\\
% &=d(x,y)



Thus, d(x,y) is scale invariant.
This example shows the case where we take the \emph{Mahalanobis distance} (while forgetting the square root) and 
do the transformation $x \rightarrow {x}' = U^{T}x$ and $y \rightarrow {y}' = U^{T}y$, so that $K = U^{T}CU$ is 
the diagonalized covariance matrix.

% subsubsection scale invariant (end)
% subsection a (end)

% section exercise_3 (end)
