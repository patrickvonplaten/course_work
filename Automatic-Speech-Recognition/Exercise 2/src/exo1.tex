\section*{Exercise 2} % (fold)
\label{sec:section_name}

\subsection*{a)} % (fold)
\label{sub:a}

\begin{enumerate}
	\item An audio speech signal $x[n]$ is fed as the input.
	\item The signal is split into distinct frames of $N$ samples and each frame is "windowed" (i.e. could be a Hamming window).
	\item Perform a Discrete fourier transform to the signal (FFT).
	\item By squaring the signal obtain power spectrum.
	\item Apply to the powers of the frequency spectrum triangular filter banks(Mel-Filters) to map signal to mel scale (also called Mel-frequency wrapping).
	\item Log every Mel-frequency magnitude to obtain the Mel cepstrum.
	\item Convert the obtained signal to cepstral domain by using the  DCT.
	\item Select the lower order coefficients as the feature vector to avoid influence from higher coefficients that contain less information. As a rule of thumb, until 13th coefficient are used, skipping first one (only 12 coefficients).
\end{enumerate}


% subsection a (end)

\subsection*{b)} % (fold)
\label{sub:b}

The cepstrum approximates the human ear system response. Human hearing perception is limited by the ear; this acts as a filter over certain frequency components, which are non-uniformly spaced on the frequency axis. This means there is more energy on the low frequency regions, and higher frequencies have a less important contribution.

% subsection b (end)
