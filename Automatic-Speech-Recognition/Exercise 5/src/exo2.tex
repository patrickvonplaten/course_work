\section*{Exercise 1c)} % (fold)
\label{sec:Exercise 1c)}

First, let's define $N_{a,b}$ to be the amount of data points between the two segments $a$ and $b$. Since, we use whole integers for a and b including both a and b, we can state: $N_{a,b} = b+1 - a$.

Expanding the term inside the cost function $h(a,b)$:

\begin{align} 
    h(a,b) &= \sum_{t=a}^{b}(x_t-\mu_{a,b})^2 \\
           &= \sum_{t=a}^{b}(x_t^2-2\mu_{a,b}x_t+\mu_{a,b}^2) \\
           &= \mu_{a,b}^2N_{a,b} + \sum_{t=a}^{b}x_t^2 -2\mu_{a,b}\sum_{t=a}^{b}x_t \\
           &= \mu_{a,b}^2N_{a,b} + \sum_{t=a}^{b}x_t^2 -2\mu_{a,b}^2N_{a,b} \\
           &= \sum_{t=a}^{b}x_t^2 -\mu_{a,b}^2N_{a,b} 
\end{align} 

By defining $S_k$ and $Q_k$ as the respective cumulative sum of the data points and cumulative sum of the squared of the data points and the mean in terms of $S_k$:
\begin{align} 
    S_k &= \sum_{t=0}^{k}x_t \\
    Q_k &= \sum_{t=0}^{k}x_t^2 \\
    \mu_{a,b} &= \frac{S_b - S_{a-1}}{N_{a,b}}
\end{align} 

Then the cost function $h(a,b)$ in terms of $S_k$ and $Q_k$ results from substitution on equation (5):

\begin{align} 
    h(a,b) &= Q_b - Q_{a-1} -\frac{(S_b - S_{a-1})^2}{N_{a,b}} \\
\end{align} 

We can see that we can now store the values of $S_k$ and $Q_k$ to avoid redoing the same calculations.
An implementation of the pseudo code would look like this: 
\begin{lstlisting}
for t in range(0, T):
    S[t] = S[t-1] + test[t]  # stores the cumulative sum of the data
    Q[t] = Q[t-1] + (test[t]**2)  # stores the cumulative sum of the square

def computeErrorFast(begin, end):
    return Q[end] - Q[begin-1] - ((S[end] - S[begin-1])**2)/(end-begin+1)
\end{lstlisting}

Here, Improvement is made in calculating local cost h, while computing an error('computeError' to 'computeErrorFast').
With it, time complexity has improved from $O(T*T)$ to linear time $O(T)$. And the entire complexity of global cost also improved accordingly. 
Memory complexity is increased by 2T by saving the intermediate cumulative values into two lists with the length of T. Thus time complexity is $O(KT^2)$. \\ 

% section Exercise 1c) (end)