\section*{2)} % (fold)
\label{sec:2_}

\subsection*{a)} % (fold)
\label{sub:a_}

Given $p(s|x)$ we are able to calculate $p(s)$.
\[
p(s) = \sum_i^N p(s|x_{i}) p(x_{i})
\]

$N$ presents here the number of data samples available for traninig.
We see that we need $p(x)$ in addition to $p(s|x)$ to calculate $p(s)$.
% subsection a_ (end)

\subsection*{b)} % (fold)
\label{sub:b_}

In practise $p(s)$ is calculate using the relative frequency, 
so we have $p(s)$ = $\frac{\text{\# of state s}}{\text{\# of all states}}$.

In practise it is
\begin{enumerate}
	\item much more time and memory consuming to calculate $p(s)$ using
the formula from a), since integration is hard for computing.
	\item We have no access for p(x) or need to be modeled separately.
\end{enumerate}

In conclusion it is much better to calculate $p(s)$ as the relative frequency,
so that we can also classify data not seen in training. 
 

% subsection b_ (end)
% section 2_ (end)