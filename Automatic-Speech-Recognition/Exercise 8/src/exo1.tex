\section*{1)} % (fold)
\newcommand{\emlog}[1]{\emph{log}}
\newcommand{\scalemath}[2]{\scalebox{#1}{\mbox{\ensuremath{\displaystyle{#2}}}}}
\label{sec:1}
HMM generating syllable sequences made up of four syllables, 'sim,'sa','la','bim'.\\
\begin{tikzpicture}[font=\small\sffamily,label=monopoly,->,>=stealth',shorten >=1pt,auto,node distance=4cm,
semithick]
  \tikzset{observ/.style={state, 
                                     minimum width=1.3cm,
                                     line width=0.13mm,
                                     fill=gray!20!white}}
  \tikzset{my_state/.style={state, 
                                     minimum width=1.3cm,
                                     line width=0.4mm,
                                     fill=gray!60!white}}
                                     
 
  
  % State
  \node[my_state, ] (Start) {0};  
  \node[my_state, right = 2 of Start] (One) {1};  
  \node[my_state, right = 2 of One] (Two) {2};  
  \node[my_state, right = 2 of Two] (Three) {3};  

  % Observations
  \node[observ, below = 4  of Start] (Sim) {sim};  	
  \node[observ, right = 2 of Sim] (Sa) {sa};  
  \node[observ, right = 2 of Sa] (La) {la};  
  \node[observ, right = 2 of La] (Bim) {bim};  
  
 % Transition Prob
 \path (Start) edge node {0.5} (One)
               edge [out=70,in=120, looseness=1] node {0.5} (Two)
       (One) edge [out=120,in=60, looseness=5] node [loop above] {0.1} (One)
             edge node {0.6} (Two)
             edge[out=70,in=120, looseness=1] node {0.3} (Three) 
       (Two) edge [out=120,in=60, looseness=5] node [loop above =16,min distance=30mm] {0.5} (Two)
             edge node {0.5} (Three)
       (Three) edge [out=120,in=60, looseness=5] node [loop above =16,min distance=30mm] {1} (Three);
        
 % EmissionProb
 \path [dashed] (One) edge [draw=blue] node [above =0.2, label={[red]}] {0.6} (Sim)
             edge [draw=blue] node [above =0.2] {0.2} (Sa)
             edge [draw=blue] node [above =0.2] {0.1} (La)
             edge [draw=blue] node [above =0.2] {0.1} (Bim);
\path [dashed] (Two) edge [draw=red] node {0.4} (Sim)
edge [draw=red] node {0.2} (Sa)
edge [draw=red] node {0.4} (La) ;
\path [dashed] (Three) edge  [draw=green] node [below left =0.6] {0.3} (Sa)
edge [draw=green] node [below =0.6] {0.3} (La)
edge [draw=green] node [below =0.6] {0.4} (Bim);             
\end{tikzpicture}

\newpage

\subsection*{a)} % (fold)
9 different paths are possible(0-1-1-1-3, 0-1-1-2-3, 0-1-1-3-3, 0-1-2-2-3, 0-1-2-3-3, 0-1-3-3-3, 0-2-2-2-3, 0-2-2-3-3, 0-2-3-3-3):
Note that the following diagram is not correct since all paths must end in state 3. So we have to delete the paths leading to "bim" at state 1 and 2.
Also note that in the first exercise, the produced sequence does not have to be "sim","sa","la","bim" but can be whatever (f.e. "sim","sim","sim","la")
So we should just use the labels (t1 - t4) for the x-axis.
 
\label{sub:a}
\begin{figure}[h!]
	\centering
	\includegraphics[width=0.8\textwidth]{./img/1a.png}
	\caption{Unfolded automaton and resulting grid}
\end{figure}



%\begin{tikzpicture}[darkstyle/.style={circle,draw,fill=gray!40,minimum size=20}]
%\foreach \i in {1,...,25}
%{
%	\pgfmathtruncatemacro{\y}{(\i - 1) / 5}
%	\pgfmathtruncatemacro{\x}{\i - 5 * \y}
%	\pgfmathtruncatemacro{\label}{\x + 5 * (4 - \y)}
%	\node[darkstyle,minimum size=20] (\label) at (1.5*\x,1.5*\y)
%	{\label};
%}
%
%% Horizontal connections
%
%\foreach \x in {1,...,4}
%\foreach \y in {0,...,4}
%{
%	\pgfmathtruncatemacro{\cur}{\x + 5* \y}
%	\pgfmathtruncatemacro{\next}{\cur + 1}
%	\draw (\cur) -- (\next);
%}
%
%% Vertical connections
%\foreach \start in {1,...,20}
%{
%	\pgfmathtruncatemacro{\down}{\start+5}
%	\draw (\start) -- (\down);  
%}
%
%\end{tikzpicture}

% subsection a (end)

\subsection*{b)} % (fold)
\label{sub:b}

$
\scalemath{0.75}{
\begin{array}
	{ | l | l | l | l | l | l | l | l | l | l | l | l | }
	\hline
	paths & s0 & sim & sa & la & bim & p(0)  & p(s1|0) \cdot p(t1|s1)  & p(s2|s1) \cdot p(t2|s2) & p(s3|s2) \cdot p(t3|s3)  & p(s4|s3) \cdot p(t4|s4)  & \  \\ \hline
	1 & 0 & 1 & 1 & 1 & 3 & 1 & 0.3 & 0.02 & 0.01 & 0.12 & 0.0000072 \\ \hline
	2 & 0 & 1 & 1 & 2 & 3 & 1 & 0.3 & 0.02 & 0.24 &	0.2 &	0.000288 \\ \hline
	3 & 0 & 1 & 1 & 3 & 3 & 1 & 0.3 & 0.02 & 0.09 & 0.4 & 0.000216 \\ \hline
	4 & 0 & 1 & 2 & 2 & 3 & 1 & 0.3 & 0.12 & 0.2 & 0.2 & 0.00144 \\ \hline
	5 & 0 & 1 & 2 & 3 & 3 & 1 & 0.3 & 0.12 & 0.15 & 0.4 & 0.00216 \\ \hline
	6 & 0 & 1 & 3 & 3 & 3 & 1 & 0.3 & 0.09 & 0.3 & 0.4 & 0.00324 \\ \hline
	7 & 0 & 2 & 2 & 2 & 3 & 1 & 0.2 & 0.1 & 0.2 & 0.2 & 0.0008 \\ \hline
	8 & 0 & 2 & 2 & 3 & 3 & 1 & 0.2 & 0.1 & 0.15 & 0.4 & 0.0012 \\ \hline
	9 & 0 & 2 & 3 & 3 & 3 & 1 & 0.2 & 0.15 & 0.3 & 0.4 & 0.0036 \\ \hline
\end{array}
}
$

\begin{align*}
\sum_\text{Possible paths}  \prod_{t=1} P(X_{t}|S_{t-1}) \cdot P(S_{t}|S_{t-1}) = 12.9 \cdot 10^{-3}
\end{align*}
% subsection b (end)

\subsection*{c)} % (fold)
\label{sub:c}
\begin{align*}	
	\text{Max} _\text{Possible paths} {\prod_{t=1}P(X_{t}|S_{t-1}) \cdot P(S_{t}|S_{t-1})} = 3.6 \cdot 10^{-3}
\end{align*}
Approximated hidden sequence of states from observation 'sim-sa-la-bim' is '0-2-3-3-3' with the approximative probability of $3.6 \cdot 10^{-3}$. 

% subsection c (end)

\subsection*{d)} % (fold)
\label{sub:d}
The big advantage of Viterbi is that we can use the \emlog~ function to simplify the equation because we only need to find the maximum value of a product, so taking 
the \emlog~ of this equation will yield the same maximum value (\emlog~ is monoton). Taking the \emlog~ of an equation with a sum does not really help to 
simplify the equation. In addition the Viterbi approximation ensures somewhat numerical stability. In the case of not using the Viterbi approximation, if the markov 
chains get longer, very low probabilities are added to the overall sum which can cause problems with ``float'' in Python for example.
The disadvantage of using the viterbi approximation is obviously that it is an approximation and thus could lead to a wrong result.
% subsection d (end)

% section 1 (end)