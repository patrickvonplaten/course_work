\section*{1)} % (fold)

In the following exercise, we will describe the pseudo code needed to solve the exercises.

\subsection*{a)} % (fold)
\label{sub:a}

The following pseude code describes an algorithm to create a phonetic prefix tree using 
an input file format as in the attached file \lstinline|lexicon.xml|.

\begin{python}
class ListOfWords():
	def __init__(self, file):
		self.listOfWords = self.makeListOfWords(file)
		#return sum of all phonemes of all list of phonemes
		self.totalAmountOfPhonemes = self.getTotalPhonemes() 
		#return count of words with len(word) <= 2
		self.totalWordsMore1Phoneme = self.countWords(2)
		#return count of words with len(word) <= 3 
		self.totalWordsMore2Phonemes = self.countWords(3) 

	def makeListOfWords(self, file):
		list = []

		# read in all distinct words containing lemmas as string
		for line in file:
			#split lines by space to a list: 'a bc d' -> [a,bc,d]
			list.append(line.split()) 
		return list
		
class Tree():
	def __init__(self):
		self.root = new Node('si')
		self.edgesForSearch = 0
		self.edgesForSeachUntil2 = 0
		self.edgesForSearchUntil3 = 0

	def insertPhonemList(self, phonemeList): 
		len = len(phonemeList)
		node = self.root

		for i in range(len):
			child = node.find(phonemeList[i])

			if(child == None):
				newChild = newNode(value)
				child.insertNode(newChild)
				node = newChild

				self.edgesForSearch += 1
				if(i < 3):
					self.edgesForSearch3 +=1 
					if(i < 2):
						self.edgesForSearch2 +=1

			else:
				node = child

class Node():
   def __init__(self, value):
   	   self.value = value
       self.children = [] 

   def insertNode(self, value):
   	   self.children.add(new Node(value));

   def find(self, value):
   	   for child in self.children: 
   	   		if(child.value == value): 
   	   			return child;
   	   return None;
\end{python}

\subsection*{b)} % (fold)
\label{sub:b}

Using the written code to 1) create the lexical prefix tree and 
2) to print out the desired compression ratio, we need to write the 
following code:

\begin{python}
if __name__ == "__main__":
	listOfWordsAsPhonemes = new ListOfWords('lexicon.xml')
	lexiconPrefixTree = new Tree()

	for phonemeList in listOfWordsAsPhonemes:
		lexiconPrefixTree.insertPhonemList(phonemeList)

	#3 since there are 3 states per word
	print('(i)', 3 * listOfWordsAsPhonemes.totalPhonemes/
		3 * lexiconPrefixTree.edgesToSearch) 
	print('(ii)', 3 * listOfWordsAsPhonemes.totalWordsMore1Phoneme/
		3 * lexiconPrefixTree.edgesForSeachUntil2)
	print('(iii)', 3 * listOfWordsAsPhonemes.totalWordsMore2Phonemes/
		3 * lexiconPrefixTree.edgesForSeachUntil3)
\end{python}

It can clearly be seen that, especially for the first two levels the the compression factor is enormous.
Even for the whole tree, we can see a compression factor of around $\simeq 2.691$.

\begin{itemize}
	\item (i) Complete tree: $\frac{241389}{89694} = 2.6913$
	\item (ii) First 2 Levels: $\frac{83142}{1932} = 43.0341$
	\item (iii) First 3 Levels: $\frac{123747}{11637} = 10.6339$
\end{itemize}


% subsection subsection_name (end)

\subsection*{c)} % (fold)
\label{sub:c_}

The difference now is that, we will have a list of triphones instead of monophones. So our list of words instead of looking like:

\begin{python}
[
	['jh','uw','l'],
	......
]
\end{python}


looks more like

\begin{python}
[
	['jh{#+uw}','uw{jh+l}','l{uw+#}'],
	......
]
\end{python}

So we need to make a new \lstinline|self.makeListOfWords| functinon:

\begin{python}
def makeListOfWords(self, file):
		list = []

		# read in all distinct words containing lemmas as string
		for line in file:
			phonemeList = []
			phonemes = line.split();

			for i in range(phonemes):
				prev = '#'
				next = '#'
				if(i > 0):
					prev = line[i-1]
				if(i < len(line) - 1)
					next = line[i+1]
				phonemeList.append(phonemes[i]+ '{' + prev + '+' next + '}')

			list.append(phonemeList)
		return list
\end{python}

Having obviously more nodes now, the compression factor will be 
much lower so that we get a compression factor of $\simeq 1.863$


\begin{itemize}
	\item (i) Complete tree: $\frac{241389}{129546} = 1.863$
	\item (ii) First 2 Levels: $\frac{83142}{12192} = 6.189$
	\item (iii) First 3 Levels: $\frac{123747}{33837} = 3.657$
\end{itemize}

% subsection c_ (end)

\subsection*{d)} % (fold)
\label{sub:d_}

Now we tied the triphones to different clusters. This strongly limits the search complexity and also makes it possible to 
assign triphones not seen in the training data to a given cluster using the CART that created the tied triphones. 

To be able to create a lexical prefix tree with the given triphone clusters, we need to modify our code since now we want to 
save the index of a given triphone cluster instead of the triphones itself as the \lstinline|node.value|. The indices of the clusters can be found in the file: 
\lstinline|cart.phone|. So instead of saving a ``list of list of phonemes'', we need to save a ``list of list of indices''.

So our modified class \lstinline|ListOfWords| looks as follows:

\begin{python}
class ListOfWords():
	def __init__(self, lexicon, phonemeCluster):
		self.phonemeIndexDict = self.makePhonemeIndexDict(phonemeCluster)
		self.listOfWords = self.makeListOfWords(lexicon)
		#return sum of all phonemes of all list of phonemes
		self.totalAmountOfPhonemes = self.getTotalPhonemes() 
		#return count of words with len(word) <= 2
		self.totalWordsMore1Phoneme = self.countWords(2)
		#return count of words with len(word) <= 3 
		self.totalWordsMore2Phonemes = self.countWords(3) 

	def makePhonemeIndexDict(self, phonemeCluster):
		phoIdxDict = {}
		for line in phonemeCluster: 
			phoIdxDict[line.split[0]] = line.split[1];

	def makeListOfWords(self, lexicon):
		list = []

		# read in all distinct words containing lemmas as string
		for line in file:
			indexList = []
			phonemes = line.split();

			for i in range(phonemes):
				prev = '#'
				next = '#'
				if(i > 0):
					prev = line[i-1]
				if(i < len(line) - 1)
					next = line[i+1]
				phoneme = phonemes[i] + '{' + prev + '+' next + '}'
				indexList.append(self.phonemeIndexDict[phoneme])

			#indexList: e.g [123, 224, 13, 25]
			list.append(indexList)
		return list
\end{python}

We would expect the compression factor to go up again since we can merge certain nodes having the same 
indices. The result is: 

\begin{itemize}
	\item (i) Complete tree: $\frac{241389}{94674} = 2.5497$
	\item (ii) First 2 Levels: $\frac{83142}{2919} = 28.4830$
	\item (iii) First 3 Levels: $\frac{123747}{13446} = 9.2032$
\end{itemize}

\subsection*{e)} % (fold)
\label{sub:e_}

Before a triphone corresponded to exactly one cluster and thus one index. Now since, we look more closely and give a different index to 
every state of a triphone. A triphone has three states. Hence an index now corresponds to the first, second or third state of multiple 
triphones. So we need to save three indexes as the value per node. For that we slightly need to change the function 
\lstinline|makeListOfWords| of class \lstinline|ListOfWords|:

\begin{python}
def makeListOfWords(self, lexicon):
	list = []

	# read in all distinct words containing lemmas as string
	for line in file:
		indexList = []
		phonemes = line.split();

		for i in range(phonemes):
			prev = '#'
			next = '#'
			if(i > 0):
				prev = line[i-1]
			if(i < len(line) - 1)
				next = line[i+1]
			phoneme = phonemes[i] + '{' + prev + '+' next + '}'

			indexList.append([self.phonemeIndexDict[phoneme + '.0'],self.phonemeIndexDict[phoneme + '.1'],
			self.phonemeIndexDict[phoneme + '.2']])

		#indexList: e.g [[123,13,569], [23,734,78], [224,13,25]]
		list.append(indexList)
	return list
\end{python}

Since the lexical prefix tree already includes the states, we don't need to multiply by 3 anymore. Instead we have to count the indices per node.

The result is a minor decrease in compression factor: 

\begin{itemize}
	\item (i) Complete tree: $\frac{241389}{107168} = 2.2524$
	\item (ii) First 2 Levels: $\frac{83142}{3270} = 15.7765$
	\item (iii) First 3 Levels: $\frac{123747}{19732} = 6.271$
\end{itemize}

% section 1 (end)