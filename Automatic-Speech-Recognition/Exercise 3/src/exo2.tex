\section*{Exercise 2} % (fold)
\label{sec:section_name}

\subsection*{a)}
\label{sub:subsection_name}

We define the vector of the reference signal as: $s \in \reals^{13} \text{ and } t \in \reals^{21}$ as the vector of the hypothesis.
Given from the design, we have: 

\begin{align} 
s =& \left[\begin{array}{ccccccccccccc} 0 & 2 & 2 & 1 & 3 & 4 & 5 & 5 & 4 & 1 & 2 & 1 & 0 \end{array}\right]^{T} \\
t =& \left[\begin{array}{ccccccccccccccccccccc} 0 & 0 & 0 & 2 & 2 & 2 & 1 & 3 & 4 & 5 & 4 & 2 & 1 & 1 & 1 & 2 & 2 & 2 & 1 & 0 \end{array}\right]^{T}	
\end{align}

In the following we will show the matrix of the local distances $M \in \reals^{13} \times \reals^{21}$, where as the matrix 
is defined as: $m_{ij} = |s_{i} - t_{i}| $.

So our final result is: 

\[ M = 
\left[
\begin{array}{cccccccccccccccccccccc}
 0 & 0 & 0 & 2 & 2 & 2 & 1 & 3 & 4 & 5 & 4 						& 3 & 2 & 1 & 1 & 1 & 2 & 2 & 2 & 1 & \fbox{0} \\
 2 & 2 & 2 & 0 & 0 & 0 & 1 & 1 & 2 & 3 & 2 						& 1 & 0 & 1 & 1 & 1 & 0 & 0 & 0 & \fbox{1} & 2 \\
 2 & 2 & 2 & 0 & 0 & 0 & 1 & 1 & 2 & 3 & 2 						& 1 & 0 & 1 & 1 & 1 & \fbox{0} & \fbox{0} & \fbox{0} & 1 & 2 \\
 1 & 1 & 1 & 1 & 1 & 1 & 0 & 2 & 3 & 4 & 3 						& 2 & 1 & \fbox{0} & \fbox{0} & \fbox{0} & 1 & 1 & 1 & 0 & 1 \\
 3 & 3 & 3 & 1 & 1 & 1 & 2 & 0 & 1 				 & 2 & 1 		& \fbox{0} & \fbox{1} & 2 & 2 & 2 & 1 & 1 & 1 & 2 & 3 \\
 4 & 4 & 4 & 2 & 2 & 2 & 3 & 1 & 0 & 1 & 			   \fbox{0}	& 1 & 2 & 3 & 3 & 3 & 2 & 2 & 2 & 3 & 4 \\
 5 & 5 & 5 & 3 & 3 & 3 & 4 & 2 & 1 & 			   \fbox{0} & 1	& 2 & 3 & 4 & 4 & 4 & 3 & 3 & 3 & 4 & 5 \\
 5 & 5 & 5 & 3 & 3 & 3 & 4 & 2 & 1 						& 0 & 1	& 2 & 3 & 4 & 4 & 4 & 3 & 3 & 3 & 4 & 5 \\
 4 & 4 & 4 & 2 & 2 & 2 & 3 & \fbox{1} & \fbox{0} 		& 1 & 0	& 1 & 2 & 3 & 3 & 3 & 2 & 2 & 2 & 3 & 4 \\
 1 & 1 & 1 & 1 & 1 & 1 & \fbox{0} & 2 & 3 & 4 				& 3	& 2 & 1 & 0 & 0 & 0 & 1 & 1 & 1 & 0 & 1 \\
 2 & 2 & 2 & \fbox{0} & \fbox{0} & \fbox{0} & 1 & 1 & 2 & 3 & 2	& 1 & 0 & 1 & 1 & 1 & 0 & 0 & 0 & 1 & 2 \\
 1 & 1 & 1 & 1 & 1 & 1 & 0 & 2 & 3 & 4 						& 3	& 2 & 1 & 0 & 0 & 0 & 1 & 1 & 1 & 0 & 1 \\
 \fbox{0} & \fbox{0} & \fbox{0} & 2 & 2 & 2 & 1 & 3 & 4 & 5 & 4 & 3 & 2 & 1 & 1 & 1 & 2 & 2 & 2 & 1 & 0 \\
\end{array}
\right]
\]

% subsection a (end)

\subsection*{b)} % (fold)
\label{sub:b}

We looking at the matrix from a), we can see the marked path in following the $\fbox{x}$.

% subsection b (end)

\subsection*{c)} % (fold)
\label{sub:c}
\subsubsection*{Recursive formula for path specification (C)}
\label{subsub:ci}
The recursion formula is derived by minimizing the total distance over both time axis $s$ and $t$:
\begin{align}
    D(t,s) &= \min_{l\to (t(l),s(l))}\left\{\sum_{l=1}^{\lambda}d(x_t(l)),y_s(l)) : (t(\lambda),s(\lambda)) = (t,s)\right\}
\end{align}

Following the logic of the recursive formula's for the path specifications from the lecture, we arrive at: 

Finally obtaining the recursive formula by considering the previous grid points and their previous distance:
\[
D(t,s) = d(x_t,y_s) + \min \{ D(t-2,s-1) + d(x_{t-1},y_s), D(t-1,s-1), D(t-1,s-2) + d(x_t,y_{s-1}) \}
\]


\subsubsection*{Recursive formula for path specification (D)}
\label{subsub:cii}

In this case, the calculation for the recursive formula requires the same first 2 steps as in previous case. Then it is just required to take the minimum over the accumulated distance sums for the 3 possible previous paths over the previous grid-points (t-1,s-2), (t-1,s-1) and (t-2,s-1). So we obtain the recursive formula taking the minimum over the previous 3 accumulated distances (i.e. each sum):

\[
D(t,s) = d(x_t,y_s) + \min \{ D(t-2,s-1), D(t-1,s-1), D(t-1,s-2)) \}
\]


% subsection c (end)

\subsection*{d)} % (fold)
\label{sub:d}

In the following we are going to draw the matrix from whereas a grid point marked with $X$ is a grid point that has to be searched 
and a grid point marked with a $O$ is a grid point that does not have to be searched for the given path specification. 

We define the lattice for the search space as $L_{search}^i$.

\subsubsection*{path specification A} % (fold)
\label{ssub:path_specification_a}

\[ L_{search}^A = 
\left[
\begin{array}{ccccccccccccccccccccc}
 X & X & X & X & X & X & X & X & X & X & X & X & X & X & X & X & X & X & X & X & X \\
 X & X & X & X & X & X & X & X & X & X & X & X & X & X & X & X & X & X & X & X & X \\
 X & X & X & X & X & X & X & X & X & X & X & X & X & X & X & X & X & X & X & X & X \\
 X & X & X & X & X & X & X & X & X & X & X & X & X & X & X & X & X & X & X & X & X \\
 X & X & X & X & X & X & X & X & X & X & X & X & X & X & X & X & X & X & X & X & X \\
 X & X & X & X & X & X & X & X & X & X & X & X & X & X & X & X & X & X & X & X & X \\
 X & X & X & X & X & X & X & X & X & X & X & X & X & X & X & X & X & X & X & X & X \\
 X & X & X & X & X & X & X & X & X & X & X & X & X & X & X & X & X & X & X & X & X \\
 X & X & X & X & X & X & X & X & X & X & X & X & X & X & X & X & X & X & X & X & X \\
 X & X & X & X & X & X & X & X & X & X & X & X & X & X & X & X & X & X & X & X & X \\
 X & X & X & X & X & X & X & X & X & X & X & X & X & X & X & X & X & X & X & X & X \\
 X & X & X & X & X & X & X & X & X & X & X & X & X & X & X & X & X & X & X & X & X \\
 X & X & X & X & X & X & X & X & X & X & X & X & X & X & X & X & X & X & X & X & X \\
\end{array}
\right]
\]

For the path specification $A$, we can see that both the movements ``one-up'' and the movement ``one-to-the-right'' are allowed, 
so we have to consider all possible gridpoints in our calculation: \textbf{$13*21=273$}.

% subsubsection path_specification_a (end)

\subsubsection*{path specification B} % (fold)
\label{ssub:path_specification_a}

\[ L_{search}^B = 
\left[
\begin{array}{cccccccccccccccccccccc}
 O & O & O & O & O & O & X & X & X & X & X & X & X & X & X & X & X & X & X & X & X \\
 O & O & O & O & O & X & X & X & X & X & X & X & X & X & X & X & X & X & X & X & O \\
 O & O & O & O & O & X & X & X & X & X & X & X & X & X & X & X & X & X & X & X & O \\
 O & O & O & O & X & X & X & X & X & X & X & X & X & X & X & X & X & X & X & O & O \\
 O & O & O & O & X & X & X & X & X & X & X & X & X & X & X & X & X & X & X & O & O \\
 O & O & O & X & X & X & X & X & X & X & X & X & X & X & X & X & X & X & O & O & O \\
 O & O & O & X & X & X & X & X & X & X & X & X & X & X & X & X & X & X & O & O & O \\
 O & O & X & X & X & X & X & X & X & X & X & X & X & X & X & X & X & O & O & O & O \\
 O & O & X & X & X & X & X & X & X & X & X & X & X & X & X & X & X & O & O & O & O \\
 O & X & X & X & X & X & X & X & X & X & X & X & X & X & X & X & O & O & O & O & O \\
 O & X & X & X & X & X & X & X & X & X & X & X & X & X & X & X & O & O & O & O & O \\
 X & X & X & X & X & X & X & X & X & X & X & X & X & X & X & O & O & O & O & O & O \\
 X & X & X & X & X & X & X & X & X & X & X & X & X & X & X & O & O & O & O & O & O \\
\end{array}
\right]
\]

For the path specification $A$, we can see that we cannot do the ``one-up'' movement anymore. Thus, our search is limited to: \textbf{$273-42-36=195$}.

% subsubsection path_specification_b (end)

\subsubsection*{path specification C} % (fold)
\label{ssub:path_specification_c}

\[ L_{search}^C = 
\left[
\begin{array}{cccccccccccccccccccccc}
 O & O & O & O & O & O & O & O & O & O & O & O & O & O & O & O & O & O & O & X & X \\ 
 O & O & O & O & O & O & O & O & O & O & O & O & O & O & O & O & O & X & X & X & X \\ 
 O & O & O & O & O & O & O & O & O & O & O & O & O & O & O & X & X & X & X & X & O \\ 
 O & O & O & O & O & O & O & O & O & O & O & O & O & X & X & X & X & X & X & O & O \\ 
 O & O & O & O & O & O & O & O & O & O & O & X & X & X & X & X & X & X & O & O & O \\ 
 O & O & O & O & O & O & O & O & O & X & X & X & X & X & X & X & O & O & O & O & O \\ 
 O & O & O & O & O & O & O & X & X & X & X & X & X & X & O & O & O & O & O & O & O \\ 
 O & O & O & O & O & X & X & X & X & X & X & X & O & O & O & O & O & O & O & O & O \\ 
 O & O & O & X & X & X & X & X & X & X & O & O & O & O & O & O & O & O & O & O & O \\ 
 O & 0 & X & X & X & X & X & X & O & O & O & O & O & O & O & O & O & O & O & O & O \\ 
 O & X & X & X & X & X & O & O & O & O & O & O & O & O & O & O & O & O & O & O & O \\ 
 X & X & X & X & O & O & O & O & O & O & O & O & O & O & O & O & O & O & O & O & O \\ 
 X & X & O & O & O & O & O & O & O & O & O & O & O & O & O & O & O & O & O & O & O \\ 
\end{array}
\right]
\]

Since, we neither have a ``one-up'' nor a ``one-to-the-right'' movement anymore, it is 
quite obvious that our search is much more limited now. There are only \todo{Count the gridpoints!} gridpoints 
left that need to be considered by our algorithm.

% subsubsection path_specification_c (end)

\subsubsection*{path specification D} % (fold)
\label{ssub:path_specification_d}

\[ L_{search}^D = 
\left[
\begin{array}{cccccccccccccccccccccc}
 O & O & O & O & O & O & O & O & O & O & O & O & O & O & O & O & O & O & O & O & X \\
 O & O & O & O & O & O & O & O & O & O & O & O & O & O & O & O & O & O & X & X & O \\
 O & O & O & O & O & O & O & O & O & O & O & O & O & O & O & O & X & X & X & X & O \\
 O & O & O & O & O & O & O & O & O & O & O & O & O & O & X & X & X & X & X & O & O \\
 O & O & O & O & O & O & O & O & O & O & O & O & X & X & X & X & X & X & O & O & O \\
 O & O & O & O & O & O & O & O & O & O & X & X & X & X & X & X & O & O & O & O & O \\
 O & O & O & O & O & O & O & O & X & X & X & X & X & X & O & O & O & O & O & O & O \\
 O & O & O & O & O & O & X & X & X & X & X & X & O & O & O & O & O & O & O & O & O \\
 O & O & O & O & X & X & X & X & X & X & O & O & O & O & O & O & O & O & O & O & O \\
 O & O & X & X & X & X & X & X & O & O & O & O & O & O & O & O & O & O & O & O & O \\
 O & X & X & X & X & X & O & O & O & O & O & O & O & O & O & O & O & O & O & O & O \\
 X & X & X & X & O & O & O & O & O & O & O & O & O & O & O & O & O & O & O & O & O \\
 X & X & O & O & O & O & O & O & O & O & O & O & O & O & O & O & O & O & O & O & O \\
\end{array}
\right]
\]

The last path specification even limits the lattice-to-be-searched a little bit more than in $C$, 
giving us only \todo{Count the gridpoints!} gridpoints left that need to be considered by our algorithm.

% subsubsection path_specification_d (end)

% subsubsection subsubsection_name (end)

% subsection d (end)

% section section_name (end)