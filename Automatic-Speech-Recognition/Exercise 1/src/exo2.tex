\section*{Exercise 2} % (fold)
\label{sec:section_name}

\subsection*{a)}
\label{sub:subsection_name}

In the case of continuous speech recognition, we have to consider all the possible combinations of words in the vocabulary $V$ for a 
given $N$-long word sequence. 
Thus, the computational complexity for hypothesizing all possible word sequences would be 
$\mathcal{O}(N) = |V|^N$.

% subsection a (end)

\subsection*{b)} % (fold)
\label{sub:b}

Assuming an isolated speech recognitino task where as the words are \emph{dependent} on each other, we get a complexity of: 
$\mathcal{O}(N) = |V|^N$.
If, on the other hand, the words are \emph{independent} on each other, the complexity can be reduced to $\mathcal{O}(N) = N|V|$.

% subsection b (end)

\subsection*{c)} % (fold)
\label{sub:c}

We can see that in the case where the words are dependent on each other, we have the same complexity for both \emph{isolated 
speech recognition} and \emph{continous speech recognition}: $\mathcal{O}(N) = |V|^N$.
The task will still be different, since we already know the word boundaries for \emph{isolated speech recognition}, where as we 
still need to find the word boundaries for \emph{continous speech recognition}.

% subsection c (end)

% section section_name (end)